\documentclass{article}
\usepackage{graphicx}
\usepackage{booktabs}
\usepackage{array}

% Define new column types
\newcolumntype{L}[1]{>{\raggedright\arraybackslash}p{#1}}
\newcolumntype{C}[1]{>{\centering\arraybackslash}p{#1}}

\begin{document}

\begin{table}[h!]
\centering
\caption{Thermal Properties of Different Layers in the Slab Model}
\resizebox{\textwidth}{!}{
\begin{tabular}{L{1.5cm} L{2cm} C{1cm} C{2cm} C{2cm} C{2cm} C{2cm}}
\toprule
\textbf{Layer No.} & \textbf{Material} & \textbf{Thickness (m)} & \textbf{Thermal Conductivity (W/mK)} & \textbf{Specific Heat (J/kgK)} & \textbf{Density (kg/m³)} & \textbf{R-value (m²K/W)} \\ \midrule
Layer 01 & Cement Mortar & 0.01 & 2.23 & 600 & 2100 & 0.0045 \\
Layer 02 & Insulation Layer & 0.03 & 0.191 & 320 & 450 & 0.1571 \\
Layer 03 & Cement Mortar & 0.01 & 2.23 & 600 & 2100 & 0.0045 \\
Layer 04 & Dense Concrete & 0.05 & 1.7 & 840 & 2200 & 0.0294 \\ \midrule
Layer 06 & Dense Concrete & 0.05 & 1.7 & 840 & 2200 & 0.0294 \\ \midrule
\multicolumn{6}{c}{\textbf{Total}} & \textbf{0.1954} \\ \bottomrule
\end{tabular}
}
\end{table}

\end{document}
